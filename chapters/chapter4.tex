\justifying
\chapter{Discussion}
\label{chapter4}

\section{Conclusions}

Our research has demonstrated the significant role of the Fast Fourier Transform (FFT) in the field of ocean rendering. The FFT, when used in conjunction with the TMA spectrum, which is grounded in empirical data, provides a remarkably efficient and persuasive approximation of stormy and calm oceanic scenes. This combination has been instrumental in achieving substantial enhancements in speed, making real-time graphics feasible, which would otherwise be unattainable with the use of the Discrete Fourier Transform (DFT) algorithm.

However, despite the efficiency of FFT-based ocean rendering, it alone is not sufficient to produce an ocean without noticeable tiling. To address this issue, we developed a technique that involves rendering multiple cascades for different wavelengths and combining them. This innovative approach effectively renders the tiling inconspicuous, thereby enhancing the visual quality of our ocean. Additionally, this method reduces computational expense as we can render multiple smaller textures instead of one large texture.

One of the most significant challenges we encountered with FFT-based ocean rendering is its limited interactivity with surroundings. This limitation could potentially affect the realism of the scene. As suggested by Jerry Tessendorf (2004) \cite{tessendorf2004}, a hybrid approach that combines FFT with other techniques could be a potential solution to this problem, and is an area worth exploring in future research.

For the shading of the ocean, we utilized a modified version of Phong shading. This method yielded satisfactory results, providing a good balance between visual quality and computational efficiency. However, to further enhance the realism of the ocean and achieve a more convincing representation, it would be prudent to consider transitioning to Physically-Based Rendering (PBR). This advanced rendering technique, which simulates the physical properties of light and materials, could potentially provide a more realistic and visually appealing depiction of the ocean.

\section{Ideas for future work}
\subsection{Enhancing Ocean Interactivity} 
Building upon the limitations identified in previous sections, a pivotal direction for future research lies in augmenting the interactivity of the ocean model. This enhancement would enable the water to dynamically respond to environmental changes such as variations in depth, the introduction of obstacles, and the influence of external forces like wakes generated by ships. 

The realization of this enhanced interactivity can be guided by the methodologies outlined in the seminal works of Tessendorf (2004) \cite{tessendorf2004} and Tessendorf (2014) \cite{tessendorf2014}. These papers provide a robust foundation for the development of advanced ocean simulation techniques, offering a promising pathway towards a more realistic and interactive ocean model.
\subsection{Incorporation of Foam Spray Representation} 
The current model's representation of the ocean surface lacks a crucial visual element: foam spray, which is typically observed during significant wave crashes. This omission highlights a limitation in the model's ability to accurately simulate real-world oceanic conditions. To address this, we propose the integration of a particle system specifically designed to simulate the foam spray effect.

A potential approach to this challenge could be the implementation of a GPU-based particle system, as suggested by Kipfer et al. (2004) \cite{kipfer2004}. This method would leverage the computational power of modern GPUs to generate a large number of particles in real-time, thereby creating a visually complex and realistic representation of foam spray.

\subsection{Transition to Physically-Based Rendering (PBR)} 
A substantial enhancement to the current model would involve transitioning from the custom Phong model to Physically-Based Rendering (PBR). This transition promises a more accurate depiction of environmental interactions, such as the ocean's response to varying light conditions like the setting sun and nocturnal environments, as well as improved specular lighting effects.

The successful implementation of PBR can be informed by the insights provided in the presentation by Mihelich and Tcheblokov (2021) \cite{mark2021}, as well as the comprehensive courses offered by Hill and McAuley (2012) \cite{stephan2012}. These resources provide valuable guidance on the principles and practicalities of PBR, offering a robust foundation.