\chapter{Discussion}
\label{chapter4}

<Everything that comes under the `Results and Discussion' criterion in the mark scheme that has not been addressed in an earlier chapter should be included in this final chapter. The following section headings are suggestions only.>

\section{Conclusions}

FFT ocean are fast and produces relistic enought oceans.\\
Having FFT algorithm is crucial and DFT wouldn't be sufficient.\\
The key of making good ocean is having spectrum that is based on empirical data.\\
To remove tiling instead of making bigger texture simulation we can simulate diffrent cascades and overlap them thus having more detail and more performance and less tilling\\
We use modified Phong Shader, however we should switch to PBR as this would give us more relistic results.


\section{Ideas for future work}
Interactive Water\\
PBR Shading\\
Buoyancy\\
Foam Spray\\
This technique produces realistic oceans in a cost-efficient way suitable for real-time rendering. However, it does not support interactive waves. For that, J. Tessendorf proposed a method called “iWaves” \cite{tessendorf2004} and a later upgraded version “eWaves” \cite{tessendorf2014}.


