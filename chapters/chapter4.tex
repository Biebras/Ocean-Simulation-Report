\justifying
\chapter{Discussion}
\label{chapter4}

\section{Conclusions}

The Fast Fourier Transform (FFT) has been instrumental in ocean rendering, providing a remarkably efficient and persuasive approximation. This efficiency is primarily attributable to the use of the JONSWAP spectrum, which is grounded in empirical data. The substantial enhancements in speed are largely credited to the FFT algorithm, without which the use of the Discrete Fourier Transform (DFT) algorithm would render real-time graphics unfeasible.

In our research, we found that the key to creating a realistic ocean lies in the selection of an empirically based spectrum, for which we employed the TMA spectrum. However, despite its efficiency, FFT-based ocean rendering is not yet sufficient to produce an ocean without tiling. To overcome this, we rendered multiple cascades for different wavelengths and superimposed them, effectively rendering the tiling inconspicuous. This approach not only removes tiling, but also enhances the quality of our ocean and reduces the computation expense as we can render multiple smaller textures.

One of the most significant challenges with FFT-based ocean rendering is its limited interactivity with surroundings. As suggested by Jerry Tessendorf in 2001[tessendorf2001], a hybrid approach could be a potential solution to this problem.

For ocean shading, we utilized a modified version of Phong shading, which yielded satisfactory results. However, to further enhance the realism of the ocean, it would be prudent to consider transitioning to Physically-Based Rendering (PBR). This transition could potentially provide a more convincing representation of the ocean.


\section{Ideas for future work}
\subsection{Enhancing Ocean Interactivity} 
Building upon the limitations identified in previous sections, a pivotal direction for future research lies in augmenting the interactivity of the ocean model. This enhancement would enable the water to dynamically respond to environmental changes such as variations in depth, the introduction of obstacles, and the influence of external forces like wakes generated by ships. 

The realization of this enhanced interactivity can be guided by the methodologies outlined in the seminal works of Tessendorf (2004) \cite{tessendorf2004} and Tessendorf (2014) \cite{tessendorf2014}. These papers provide a robust foundation for the development of advanced ocean simulation techniques, offering a promising pathway towards a more realistic and interactive ocean model.
\subsection{Incorporation of Foam Spray Representation} 
The current model's representation of the ocean surface lacks a crucial visual element: foam spray, which is typically observed during significant wave crashes. This omission highlights a limitation in the model's ability to accurately simulate real-world oceanic conditions. To address this, we propose the integration of a particle system specifically designed to simulate the foam spray effect.

A potential approach to this challenge could be the implementation of a GPU-based particle system, as suggested by Kipfer et al. (2004) \cite{kipfer2004}. This method would leverage the computational power of modern GPUs to generate a large number of particles in real-time, thereby creating a visually complex and realistic representation of foam spray.

\subsection{Transition to Physically-Based Rendering (PBR)} 
A substantial enhancement to the current model would involve transitioning from the custom Phong model to Physically-Based Rendering (PBR). This transition promises a more accurate depiction of environmental interactions, such as the ocean's response to varying light conditions like the setting sun and nocturnal environments, as well as improved specular lighting effects.

The successful implementation of PBR can be informed by the insights provided in the presentation by Mihelich and Tcheblokov (2021) \cite{mark2021}, as well as the comprehensive courses offered by Hill and McAuley (2012) \cite{stephan2012}. These resources provide valuable guidance on the principles and practicalities of PBR, offering a robust foundation.