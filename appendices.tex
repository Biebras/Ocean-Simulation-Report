\begin{appendices}

%
% The first appendix must be "Self-appraisal".
%
\chapter{Self-appraisal}

\section{Critical self-evaluation}

In the initial stages, a comprehensive evaluation of multiple libraries was conducted to select the most suitable one for our project. This selection process was underpinned by a thorough understanding of the theoretical aspects of FFT waves, which we developed by reviewing numerous research papers. To further solidify our foundational knowledge, we created a naive sum of sins project to understand the basics of ocean generation.

With a detailed workflow and a graphed algorithm in place, we embarked on the project, ensuring that unnecessary refactoring was minimized and adherence to the project timeline was maintained. The project was initially implemented in a simplistic manner to grasp the underlying theory. Once a solid understanding was established, the project was advanced to a more complex level to optimize computational efficiency. Throughout this process, data visualization was employed at every step to ensure the absence of bugs and validate the accuracy of our results.

Upon completion of the project, the results produced were compared with real-world data. This comparison confirmed the realism of the ocean generated by our project, attesting to the success of our approach. Thus, the project execution demonstrated a high degree of planning, theoretical understanding, practical implementation, and validation, resulting in a realistic and efficient ocean generation technique.

\section{Personal reflection and lessons learned}

This project served as a comprehensive learning experience, enhancing our technical skills, academic research abilities, professional communication, and project management strategies. We learned the importance of clear and timely communication, particularly with superiors, when the project was initially undertaken in Unity without prior approval. This oversight was later rectified by submitting a detailed request analyzing different approaches.

The project also provided an opportunity to delve into academic research, enhancing our ability to effectively search for and read research papers. On the technical front, we gained experience in using compute shaders and expressing complex mathematical equations, which was instrumental in creating the FFT algorithm and generating a realistic ocean.

Additionally, the project honed our report writing skills and reinforced the necessity of obtaining necessary permissions before embarking on significant project decisions.

\section{Legal, social, ethical and professional issues}

\subsection{Legal issues}
In the context of this project, the primary legal considerations pertain to the use of Unity, a third-party tool. As the entirety of the codebase was authored independently, there are no concerns regarding the infringement of external code licenses. However, it is crucial to adhere to the terms and conditions stipulated by Unity’s license agreement. This adherence ensures the lawful utilization of Unity’s resources and capabilities, thereby aligning the project with the requisite legal standards.

\subsection{Social issues}
The potential applications of this project, particularly in domains such as video games or cinematic productions, could have notable social implications. Specifically, the manner in which the ocean is represented and rendered using this implementation could influence societal perceptions of marine environments. As such, it is crucial to ensure that the oceanic simulations generated are as accurate and realistic as possible, to foster an authentic understanding and appreciation of our oceans.

\subsection{Ethical issues}
In the realm of ethical considerations, it is imperative to ensure that the ocean simulation does not misrepresent or oversimplify the inherently complex marine phenomena. Such misrepresentations could potentially lead to misunderstandings or misuse of the work, thereby violating ethical guidelines of accuracy and truthfulness in scientific representation.

Furthermore, the consideration of making this project open-source aligns with the ethical principle of knowledge sharing in the academic and scientific community. By doing so, the project could serve as a learning resource for others, promoting transparency, collaboration, and collective learning.

\subsection{Professional issues}
A key professional consideration in this project is the adherence to industry standards and best practices in coding and documentation. This adherence ensures the maintainability, readability, and scalability of the code, thereby enhancing its longevity and usability.

Furthermore, it is crucial to stay abreast of the latest advancements in the field. This includes keeping up-to-date with new versions or features of Unity, advancements in FFT algorithms, or GPU programming techniques. Such continual learning and adaptation are essential for maintaining the relevance and effectiveness of our work in a rapidly evolving field.
%
% Any other appendices you wish to use should come after "Self-appraisal". You can have as many appendices as you like.
%
\chapter{External Material}
%<This appendix should provide a brief record of materials used in the solution that are not the student's own work. Such materials might be pieces of codes made available from a research group/company or from the internet, datasets prepared by external users or any preliminary materials/drafts/notes provided by a supervisor. It should be clear what was used as ready-made components and what was developed as part of the project. This appendix should be included even if no external materials were used, in which case a statement to that effect is all that is required.>
\section{Skyboxes}
Additional skyboxes utilized in this project were procured from the Unity Asset Store. The specific asset employed is freely available and grants comprehensive permissions for unrestricted usage. This approach aligns with the principles of ethical use of third-party resources in academic projects. The asset can be accessed at the following URL: \url{https://assetstore.unity.com/packages/2d/textures-materials/sky/skybox-series-free-103633}.

%
% Other appendices can be added here following the same pattern as above.
%
\chapter{Additional Results}
The outcomes of the conducted experiments, encapsulated in video format, are accessible in the designated \href{https://github.com/uol-feps-soc-comp3931-2324-classroom/final-year-project-Biebras}{Git repository}.

\end{appendices}
